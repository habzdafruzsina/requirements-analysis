\documentclass{article}
\usepackage{graphicx} % Required for inserting images
\usepackage{draftwatermark}
\usepackage[utf8]{inputenc}
\usepackage[english]{babel}

\SetWatermarkText{Minta}
\SetWatermarkFontSize{4cm}
\SetWatermarkScale{5}
\sloppy

\title{Projektcím}
\author{írok}
\date{November 2023}

\begin{document}

\maketitle

\section{Revision History}
(ki mit csinált, mondjuk ezt a részét kevéssé értem...)

\maketitle

\section{Projekt bemutatása}
(összefoglaló ezekről, kb 1 oldal;
miről szól a projekt?
kinek a számára készül?
ki a kivitelező?
mik a kritériumok?
miért készül a projekt, miért van szükség rá?
mi a célja?
időtartam, költségvetés
rizikók)


\begin{itemize}
\item felsorolás1
\item felsorolás2
\end{itemize}

\maketitle

\section{Követelmények}

\subsection{XY1 Követelménycsoport}


- követelmények számozása és felépítése
(hogyan épül fel egy azonosító; követelménycsoport-id)

- a megvalósítás fázisai
(fázisok felsorolása, rövid leírása, követelményeik, időtartama)

- XY1 követelmény csoport követelményei
(követelmények táblázatos leírása, azonosító, leírás)

- xY2 ...


\begin{tabular}{ |c|c| }
    \hline
    Azonosító & Leírás \\
    \hline\hline
    azonosító1 & leírás1....... \\
    \hline
    azonosító2 & leírás2....... \\
    \hline
\end{tabular}



\maketitle

\section{Szójegyzék}

(szavak csoportosítva, definíciókkal ellátva)



\end{document}